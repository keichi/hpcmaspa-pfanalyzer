\documentclass[10pt, a4paper]{IEEEtran}

% Package imports
\usepackage[style=ieee,citestyle=numeric]{biblatex}
\usepackage[english]{babel}
\usepackage{csquotes}
\usepackage{enumitem}
\usepackage{fancyvrb}
\usepackage{flushend}
\usepackage{graphicx}
\usepackage{listings}
\usepackage{longtable}
\usepackage{newclude}
\usepackage{newtxtext}
\usepackage{newtxmath}
\usepackage[format=hang]{subfig}
\usepackage{url}
\usepackage[hidelinks]{hyperref}
\RequirePackage[l2tabu, orthodox]{nag}

% Fix for pandoc 1.14
\providecommand{\tightlist}{}

\graphicspath{{./img/}}

% Source code
\lstset{%
  language={C},
  basicstyle={\small\ttfamily},%
  identifierstyle={\small\ttfamily},%
  commentstyle={\small\itshape},%
  keywordstyle={\small\bfseries},%
  ndkeywordstyle={\small\ttfamily},%
  stringstyle={\small\ttfamily},
  frame={tb},
  breaklines=true,
  columns=[l]{fullflexible},%
  numbers=none,%
  xrightmargin=0em,%
  xleftmargin=0em,%
  numberstyle={\scriptsize},%
  stepnumber=1,
  numbersep=1em,%
  lineskip=-0.5ex,%
  mathescape%
}

\addbibresource{references.bib}

\begin{document}

\title{Performance Simulation and Evaluation of SDN-enhanced MPI}

\author{%
    \IEEEauthorblockN{%
        Keichi Takahashi\IEEEauthorrefmark{1},
        Susumu Date\IEEEauthorrefmark{1},
        Dashdavaa Khureltulga\IEEEauthorrefmark{1},
        Yoshiyuki Kido\IEEEauthorrefmark{1}, \\
        Hiroaki Yamanaka\IEEEauthorrefmark{2},
        Eiji Kawai\IEEEauthorrefmark{2},
        Shinji Shimojo\IEEEauthorrefmark{1}
    } \\
    \IEEEauthorblockA{%
        \IEEEauthorrefmark{1}
        Osaka University\\
        Osaka, Japan\\
        \{takahashi.keichi, huchka\}@ais.cmc.osaka-u.ac.jp,
        \{date, kido, shimojo\}@cmc.osaka-u.ac.jp
    } \\
    \IEEEauthorblockA{%
        \IEEEauthorrefmark{2}
        National Institute of Information and Communications Technology\\
        Tokyo, Japan\\
        \{hyamanaka, eiji-ka\}@nict.go.jp
    }
}

\maketitle

% Abstract
\begin{abstract}
    Due to the rapid scale out of high-performance computing systems,
    interconnects are becoming increasingly large-scale and complex. This
    trend is making static and over-provisioned interconnects
    cost-ineffective. We have been developing SDN-enhanced MPI, a framework
    that optimizes the interconnect to fit the communication patterns of MPI
    applications by leveraging the dynamic network controllability of
    Software-Defined Networking (SDN). Our previous works have demonstrated
    the acceleration of several individual MPI primitives based on the idea of
    SDN-enhanced MPI\@. However, the effect of SDN-enhanced MPI on the
    utilization of interconnects and performance of real-world applications is
    yet unclear. To answer this question, we developed an MPI tracer and
    online analyzer to extract the communication patterns from applications.
    Furthermore, we developed an interconnect simulator to investigate the
    congestion in the interconnect for a given cluster configuration and set
    of communication patterns. By using these tools, we show how SDN-enhanced
    MPI can reduce congestion in the interconnect and potentially improve the
    performance of applications.
\end{abstract}

\begin{IEEEkeywords}
    Message Passing Interface, Software Defined Network, OpenFlow, Interconnects
\end{IEEEkeywords}

% Body
\include*{src/1_intro}
\include*{src/2_goal}
\include*{src/3_proposal}
\include*{src/4_evaluation}
\include*{src/5_literature_review}
\include*{src/6_conclusion_and_future_work}

\section*{Acknowledgement}
This work was supported by JSPS KAKENHI Grant Number JP26330145. This research
was partly supported by collaborative research of the National Institute of
Information and Communication Technology and Osaka University (Research on
high functional network platform technology for large-scale distributed
computing).


% Bibliography
\printbibliography[heading=bibintoc,title={References}]

% Fix bibliography indentation issue
\par\leavevmode

\end{document}
