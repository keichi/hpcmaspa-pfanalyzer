\documentclass[conference]{IEEEtran}

% Package imports
\usepackage[style=ieee,citestyle=numeric]{biblatex}
\usepackage[english]{babel}
\usepackage{csquotes}
\usepackage{enumitem}
\usepackage{fancyvrb}
% \usepackage{flushend}
\usepackage{graphicx}
\usepackage{listings}
\usepackage{longtable}
\usepackage{newclude}
\usepackage{newtxtext}
\usepackage{newtxmath}
\usepackage[labelformat=simple]{subcaption}
\usepackage{url}
\usepackage[hidelinks]{hyperref}
\RequirePackage[l2tabu, orthodox]{nag}

\renewcommand\thesubfigure{(\alph{subfigure})}

% Fix for pandoc 1.14
\providecommand{\tightlist}{}

\graphicspath{{./img/}}

% Source code
\lstset{%
  language={C},
  basicstyle={\small\ttfamily},%
  identifierstyle={\small\ttfamily},%
  commentstyle={\small\itshape},%
  keywordstyle={\small\bfseries},%
  ndkeywordstyle={\small\ttfamily},%
  stringstyle={\small\ttfamily},
  frame={tb},
  breaklines=true,
  columns=[l]{fullflexible},%
  numbers=none,%
  xrightmargin=0em,%
  xleftmargin=0em,%
  numberstyle={\scriptsize},%
  stepnumber=1,
  numbersep=1em,%
  lineskip=-0.5ex,%
  mathescape%
}

\addbibresource{references.bib}

\begin{document}

\title{A Toolset for Analyzing Application-aware Dynamic Interconnects}

\author{%
    \IEEEauthorblockN{%
        Keichi Takahashi\IEEEauthorrefmark{1},
        Susumu Date\IEEEauthorrefmark{1},
        Dashdavaa Khureltulga\IEEEauthorrefmark{1},
        Yoshiyuki Kido\IEEEauthorrefmark{1}, \\
        Shinji Shimojo\IEEEauthorrefmark{1}
    } \\
    \IEEEauthorblockA{%
        \IEEEauthorrefmark{1}
        Osaka University\\
        Osaka, Japan\\
        \{takahashi.keichi, huchka\}@ais.cmc.osaka-u.ac.jp,
        \{date, kido, shimojo\}@cmc.osaka-u.ac.jp
    } \\
}

\maketitle

% Abstract
\begin{abstract}
    Recent rapid scale out of high performance computing systems has
    been continuously increasing the scale and complexity of the
    interconnects. As a result, current static and over-provisioned
    interconnects are becoming cost-ineffective. From the background, we have
    been currently working on the integration of network programmability into
    the interconnect control, based on our belief that dynamically optimizing
    the packet flow in the interconnect to fit the communication pattern of
    applications can increase the utilization of interconnects and improve
    application performance. Simulators come in handy when investigating the
    performance characteristics of interconnects with different topologies and
    parameters. However, little effort has been put for the simulation of
    dynamically controlled interconnects, while simulators for static
    interconnects has been extensively researched and developed. To facilitate
    the analysis on the performance characteristics of dynamic interconnects,
    we have developed a simulator specialized for dynamic interconnects. Our
    simulator allows interconnect researchers and designers to investigate the
    congestion in the interconnect for an arbitrary cluster configuration and
    a set of communication patterns collected by a dedicated tool. Through the
    use of the toolset, this paper shows how dynamically controlling the
    interconnects can reduce congestion and potentially improve the
    performance of applications.
\end{abstract}

\begin{IEEEkeywords}
    Simulation, Profiling, Interconnect, Message Passing Interface, Software
    Defined Network
\end{IEEEkeywords}

% Body
\include*{src/1_intro}
\include*{src/2_goal}
\include*{src/3_proposal}
\include*{src/4_evaluation}
\include*{src/5_literature_review}
\include*{src/6_conclusion_and_future_work}

\section*{Acknowledgement}
This work was supported by JSPS KAKENHI Grant Number JP26330145. This research
was supported in part by ``Program for Leading Graduate Schools'' of the
Ministry of Education, Culture, Sports, Science and Technology, Japan.

% Bibliography
\printbibliography[heading=bibintoc,title={References}]

% Fix bibliography indentation issue
\par\leavevmode

\end{document}
